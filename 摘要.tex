\chapter*{摘要}
\markboth{摘要}{}
\addcontentsline{toc}{chapter}{摘要}
斯格明子是一种具有类似粒子特性且具有拓扑保护性质的自旋结构,它于2009年首次在实验中被观察到。
由于它的高稳定性、纳米级尺寸以及低驱动电流密度等特点,斯格明子被认为可以作为信息载体并应用于下一代自旋电子器件,例如赛道存储器、自旋纳米振荡器和二极管等。
近年来,一些实验已经证明在室温下使用自旋极化电流可以形成和驱动斯格明子,这增强了斯格明子作为信息载体的应用前景。\par
斯格明子最有前途的应用之一是赛道存储器。
然而,一个斯格明子只能代表一个二进制位信息,并且斯格明子之间的排斥力使它们间距相对较大,这不利于高密度信息存储。
斯格明子袋可以很好地解决这个问题,它由一个斯格明子和任意数量的内部反斯格明子构成,而不同内部反斯格明子的数量可以代表不同的二进制位信息。
研究表明,斯格明子袋可以像斯格明子一样在自旋电流的驱动下沿着纳米轨道移动并在通过缺陷时能够表现出良好的拓扑保护性,这更有利于作为信息载体并应用于高密度赛道存储器。\par
本论文通过微磁学模拟的方法,研究了在合成反铁磁赛道上电流驱动不同拓扑度的斯格明子通过压控磁各向异性(VCMA)电压门的动力学。
首先,研究了斯格明子通过电压门时速度的变化,以及电压门产生的垂直磁各向异性(PMA)梯度与驱动电流密度和斯格明子最小速度的关系;
其次,探讨了材料参数对Co/Pt薄膜磁化结构的影响, 确认了斯格明子袋能够稳定存在的参数范围;
最后,研究了斯格明子袋在电流的驱动下通过电压门控制的PMA梯度时发生的不同拓扑结构转变。
主要研究结果如下:
\begin{enumerate}[label=(\arabic*), leftmargin=4ex, labelsep=1.4ex, itemsep=0pt, topsep=1ex,partopsep=1ex,parsep=0pt]
    \item 斯格明子正向通过电压门的临界电流密度远小于它负向通过相同的电压门的临界电流密度,这是斯格明子二极管的基本原理。
    斯格明子正向和负向通过不同的电压门时其最小速度和驱动电流密度具有线性相关性,这是斯格明子二极管的特性,类似于电子二极管的伏安特性。
    \item 斯格明子的各向异性能量随位置的变化曲线具有可逆性,即该曲线可以同时表现出斯格明子正向和负向进入电压门的过程。
    曲线斜率的大小可以近似表示斯格明子通过电压门时受到的平均阻力大小,曲线斜率的绝对值越大表示斯格明子受到的平均阻力越大。
    \item 斯格明子袋S(4)通过不同电压门时会发生不同拓扑结构的转变。
    通过控制电压门内的PMA梯度和驱动电流密度可以实现从S(4)到S(N) (N = 0 - 4)的转变,这本质上是斯格明子袋内反斯格明子之间排斥力变化的结果。
    \item 通过在一个赛道上设置三个电压门可以实现从S(4)到S(1)的连续拓扑转变,即每个电压门只阻止一个内部反斯格明子通过。
    通过计算,每个内部反斯格明子的能量在\mySI{e-19}{J}量级。
\end{enumerate}
\vspace{1em}
\textbf{关键词}:斯格明子;斯格明子袋;合成反铁磁赛道;压控磁各向异性;微磁模拟
\newpage
\thispagestyle{empty}