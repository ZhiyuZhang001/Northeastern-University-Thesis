\chapter*{Abstract}
\markboth{Abstract}{}
\addcontentsline{toc}{chapter}{Abstract}
Magnetic skyrmions, as topologically protected magnetization configurations with particle-like properties, were first experimentally observed in 2009. 
Due to their stability, nanoscale size and low driving current density, skyrmions are considered as promising candidates in the next generation of spintronic devices, such as racetrack memories, nano-oscillators, and diodes. 
In recent years, some experiments have revealed that skyrmions can be generated and driven by electric current at room temperature, which enhances the application prospects of skyrmions as information carriers. \par
One of the most promising applications for skyrmion is racetrack memories. 
However, one skyrmion can only represent one bit and the repulsion between skyrmions keeps them at a relatively long distance, which is not conducive to high-density information storage. 
An elegant solution to this could be the skyrmion bag, which are nested skyrmionic structures with high degree of freedom of topological charge numbers. 
Skyrmion bags are composed of a single skyrmion outer boundary and a number of inner antiskyrmions, which enables them to represent different information with different topological charges. 
In recent research, skyrmion bags can move along nanotracks driven by spin current, just like skyrmions. 
In addition, skyrmion bags exhibit good topological protection when passing through defects, which makes them have greater advantages in application to the racetrack memory.\par
In this paper, we investigate the dynamics of the skyrmion with different topological charges driven by spin current on synthetic antiferromagnetic racetracks with a voltage-controlled magnetic anisotropy (VCMA) gate using micromagnetic simulations. 
To begin with, we study the variation of velocity when skyrmion passes through different voltage gates, and the minimum velocity of skyrmion as a function of the perpendicular magnetic anisotropy (PMA) gradient and the driving current density. 
Furthermore, we explore the effect of material parameters on the magnetization structure of Co/Pt films, and confirm the parameter range in which the skyrmion bag can exist stably. 
Finally, we investigate the different topological transitions of skyrmion bag as it passes through voltage gates driven by a current. 
The main research results are as follows:
\begin{enumerate}[label=(\arabic*), leftmargin=4ex, labelsep=1.4ex, itemsep=0pt, topsep=1ex,partopsep=1ex,parsep=0pt]
    \item The critical current required for skyrmion to positively pass through a voltage gate where the PMA constant increases gradually, is much less than that for skyrmion to negatively pass through the voltage gate where the PMA constant decreases gradually. 
    When the skyrmion passes through different voltage gates in the positive and negative direction, the minimum velocity ($V_{min}$) of skyrmion is linearly correlated with the driving current density ($J_{spin}$). 
    The $J_{spin}$--$V_{min}$ characteristics of the skyrmion diode are similar to the volt -- ampere characteristics of the electronic diode.
    \item The variation curves of the PMA energy with the position of skyrmion are very similar when the skyrmion passes through the same voltage gate positively and negatively. 
    In other words, a variation curve can contain both the stage of positive entry of skyrmion into the voltage gate and the negative entry of the skyrmion into the voltage gate. 
    The average resistance of the skyrmion passing through the voltage gate can be seen from the slope of the curve, and a larger absolute value of the slope of the curve means that the skyrmion is subjected to a higher average resistance.
    \item Skyrmion bag S(4) undergoes different topological transitions when it passes through different voltage gates. 
    The transition from S(4) to S(N) (N = 0 - 4) can be achieved by controlling the PMA gradient in the voltage gate and driving current density. 
    The different topological transitions of skyrmion bag are related to the repulsive forces between the inner antiskyrmions.
    \item A continuous topological transition from S(4) to S(1) can be achieved by placing three voltage gates on a racetrack, i.e., each voltage gate prevents only one inner antiskyrmion from passing. 
    By calculation, the energy of each inner antiskyrmion is on the order of \SI{e-19}{J}.
\end{enumerate}
\vspace{1em}
\textbf{Keywords}:skyrmion, skyrmion bag, synthetic antiferromagnetic racetrack, voltage-controlled magnetic anisotropy, micromagnetic simulation
\newpage