\chapter{微磁学原理及数值计算方法}
微磁学模拟由于其出色的预测能力和对磁化动力学的表征及对实验结果的深入解释,已经成为纳米磁学研究中广泛使用的工具。
在过去的二十年中,微磁学模拟从只能由高度专业化的团队实现,发展到现在已经有了多种开源和商业的软件包,使得非专家也能运行自己的模拟。
微磁学的起源可以追溯到20世纪60年代,当时Brown建立了微磁学的理论基础,随后Labonte发表了他的薄膜中畴壁的内部结构的相关工作。
从一开始就显而易见的是,为了获得物理上有意义的结果,许多问题需要模拟大量参数,因此多年来对加速模拟的不同方法的探索一直没有停止。
迄今为止,大部分的努力都集中在开发更高效计算静磁场的方法上,因为这是微磁模拟中耗时最多的部分。
此外,在实现复杂的时间积分算法方面也取得了一些进展,从而允许使用更大的时长并能保持数值稳定性。\par
另一方面,在过去三十年中,显微技术的显著进步使人们可以制备和测量更小尺寸的样品,甚至可以研究纳米尺度的材料结构和性质。
显微技术的巨大进步与计算机速度和容量的显著改善以及先前提到的更高效的计算方法的发展并驾齐驱。
因此,随着时间的推移,使用个人电脑在合理的时间成本内进行具有真实器件尺寸的模拟变得越来越可行,这是微磁模拟变得流行的一个关键因素。
\section{微磁学基本能量}
磁性系统的能量可以通过哈密顿量来计算,它由不同的能量项组成。
此外,影响Landau-Lifshitz-Gilbert (LLG)方程中磁化动态的有效场也可以从系统的哈密顿量中计算出来。
本节介绍了几种微磁学中常用的能量项,并展示了它们对磁矩的影响。
\subsection{Zeeman能}
Zeeman能倾向于将所有磁矩与外磁场$H_{ext}$平行排列,其能量密度可以表示为:
\subsection{单轴各向异性能}
\subsection{交换能}
\subsection{Dzyaloshinskii-Moriya相互作用(DMI)能}
\subsection{Ruderman-Kittel-Kasuya-Yosida (RKKY)能}
\subsection{退磁能}
\section{磁化动力学方程}
\subsection{Landau-Lifshitz-Gilbert (LLG)方程}
\subsection{自旋转移力矩(STT)}
\section{微磁模拟软件(Ubermag)}
\subsection{微磁建模(Micromagneticmodel)}
\subsection{微磁计算求解器(OOMMFC)}	
\subsection{数据分析(Micromagneticdata)}
\subsection{开源性}
\newpage